\documentclass{article}
\usepackage{amsmath}
\usepackage{amssymb}

\begin{document}
\title{project ideas}
\author{howon lee}
\maketitle
\section{Idea One: Programmer Productivity}
Programmer productivity is usually hard to measure: if you look at lines of code, github commits, bugfixes done per day, average age of open issues, they are very very easy to game. But this also seems like a domain where very much can be learned, and perhaps a structure in the data could be found which would indicate programmer productivity while being harder to game than existing empirical-numerical measures. Because we also want to define who the good programmer is, this may be a semi-supervised ML task. There is a great datasource in github's data logs: they are archived in a data analytics-friendly format in githubarchive.org. The learned function would take in an individual programmer's git logs(git logs because githubarchive has that data: phabricator and JIRA data would be nice but is harder to get) and pop out a rating on several dimensions and an overall measure of how productive they've been. There may be some rule-based stuff to prevent gaming, too. It seems upon a lit search that scholarly thought about programmer productivity has mostly been non-computational after the downfall of programmer-measuring generally.

\section{Idea Two: Recommender Systems for the Musician}
I did a small lit search and I couldn't find anybody talking about recommender systems that recommend something for the musician to do in a performance. That is, a recommender system that pops up at a screen that a musician is looking at while making music and suggests a note or a musical phrase or something to play. Especially in structured but improvisational genres, this might be useful. People have done purely aleatoric (purely random) music before (some of them actual honest-to-god musicians, like John Cage), but I couldn't find anyone structuring the probability using machine learning techniques. This seems cool for the coolness of it and for the engineering challenge of making a really nice low-latency recommender system. There are plenty of datasources of MIDI music which would be suitable to such a thing, and music21 or something like it would be used for dealing with the music itself, along with a generative model. The learned function would take in the current state of the musician's performance (performance probably on a computer keyboard using stdin instead of a proper MIDI keyboard, since I'm cheap) and generate possible next notes.

The other most interesting problem is how to deal with polyphony and dynamics, too, because it seems too easy to merely put data in a markov chain or something to deal with harmony: I have to go further than that, probably with other generative models.

\end{document}